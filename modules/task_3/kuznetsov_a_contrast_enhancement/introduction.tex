\section*{Введение}
\addcontentsline{toc}{section}{Введение}
Один из наиболее распространенных дефектов фотографических, сканерных и телевизионных изображений – слабый контраст. Дефект во многом обусловлен ограниченностью диапазона воспроизводимых яркостей. Под контрастом понимается разность максимального и минимального значений яркости. Контрастность изображения можно повысить за счет изменения яркости каждого элемента изображения и увеличения диапазона яркостей. Существует несколько методов, основанных на вычислении гистограммы. В данной лабораторной работе будет рассматриваться алгоритм \textbf{линейного растяжения гистограммы}.