\sloppy
\usepackage[utf8]{inputenc}
\usepackage[T2A]{fontenc}
\usepackage{babel}
\usepackage[a4paper,
left=2cm,
right=2cm,
top=2cm,
bottom=2cm]{geometry}
\usepackage{color}
\usepackage{mathtools}
\usepackage{listings}
\usepackage{graphicx}
\usepackage{tocloft}
\usepackage{indentfirst}
\usepackage{enumitem}
\usepackage{hyperref}

\setlength{\parindent}{0.8cm}
\setlength{\parskip}{0.4cm}
\renewcommand{\contentsname}{}
\renewcommand{\cftsecleader}{\cftdotfill{\cftdotsep}}
\renewcommand\refname{Список литературы}
\definecolor{dkgreen}{rgb}{0,0.6,0}
\definecolor{gray}{rgb}{0.5,0.5,0.5}
\definecolor{mauve}{rgb}{0.58,0,0.82}

\usepackage{caption} %заголовки плавающих объектов
\captionsetup[figure]{name=Рис} 

%Псевдокодик
\usepackage{algorithm}
\usepackage{algpseudocode}
\floatname{algorithm}{Алгоритм}

%C++ code 
\usepackage{listings}
\usepackage{color}
% параметры стиля
\lstdefinestyle{mystyle}{
	basicstyle=\footnotesize,
	breakatwhitespace=false, 
	breaklines=true,
	captionpos=b,
	keepspaces=true, 
	numbers=left,
	numbersep=5pt,  % номера строк
	showspaces=false, % не показывать пробелы
	showstringspaces=false,
	showtabs=false,                  
	tabsize=2
}

% применяем стиль
\lstset{style=mystyle}
% используем заданный нами моноширинный шрифт
\lstset{basicstyle=\footnotesize\ttfamily,breaklines=true}
